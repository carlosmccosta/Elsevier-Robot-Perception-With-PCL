\section{Conclusions}\label{sec:conclusions}

The proposed localization system is able to maintain pose tracking with less than 1-2 centimeters of translation error and less than a 1-3 degrees of rotation error (in 3 and 6 DoF respectively) with the robot / sensors moving at several velocities even in cluttered and dynamic environments. Moreover, when tracking is lost or no initial pose is given, the system is able to find a valid global pose estimate by switching to more robust registration algorithms that use feature matching. This approach achieved fast pose estimation with robust tracking recovery and reliable initial pose estimation while also providing the set of the accepted initial poses before registration refinement, which can be very valuable information for a navigation supervisor when the robot is in an ambiguous region that can be registered in similar zones of the known map. The system also allows dynamic reconfiguration of the number of laser scans to assemble in order to mitigate laser measurement errors and can adapt its rate of operation according to the robot estimated velocity.

The sub-centimeter accuracy achieved by the proposed localization system along with the dynamic map update capability (using partial or full sensor data integration) and the need of no artificial landmarks / ambient modifications will allow the fast deployment of mobile robots capable to operate safely and accurately in cluttered environments.
